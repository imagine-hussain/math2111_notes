\newcommand\barbelow[1]{\stackunder[1.2pt]{$#1$}{\rule{.8ex}{.075ex}}}

% TODO: #3 Integration Notes
\section{Integration}

\subsection{Riemann and Fubini}

\paragraph{Riemann Integral}

For a bounded function \(f: R \to \mathbb{R}\), if there exists a unique
number \(I\) such that
\[
    \underline{S}_{\mathcal{P}_1, \mathcal{P}_2}(f) \leq I \leq \bar{S}_{\mathcal{P}_1, \mathcal{P}_2}(f)
\]

for every pair of partitions \(\mathcal{P}_1, \mathcal{P}_2\). Then, \(f\) is Riemann
integrable on \(R\) and
\[I = \int\int_R f = \int\int_R f(x, y) dA.\]
\(I\) is called the Riemann integral of \(f\) over\(R\).

\paragraph{Properties of the Riemann Integral}
The single variable interpretation of a Riemann integral is the (signed) area bound by the 
graph \(y = f(x)\) and the \(x\) axis over the interval \([a, b]\).
For two variables, \(\int\int_R f\) is the (signed) volume bounded by the graph
\(z = f(x, y)\) and the \(xy\)-plane over the rectangle \(R\).

If \(f, g\) are integrable on \(R\),
\begin{itemize}
    \item \(\int\int_R \alpha f = \beta g = \alpha \int\int_r f + \beta \int\int_R g\)
    \item If \(f(x)\leq g(x), \forall x\in R\) then \(\int\int_R \leq \int\int_R g\)
    \item \(\left| \int\int_R f \right| \leq \int\int_R \left| f \right | \)
    \item If \(R = R_1 \cup R_2\) and \(\text{interior }R_1 \cap \text{interior }R_2 = \emptyset\) then
    \(\int\int_R f = \int\int_{R_1} f + \int\int_{R_2} f\)
\end{itemize}

\paragraph{Fubini's Theorem}
Let \(f: R \to \mathbb{R}\) be continuous on a rectangular domain \(r = [a, b]\times [c, d]\).
Then, \(\int_a^b \int_c^d f(x, y) dy dx = \int_c^d \int_a^b f(x, y) dy, dy\).

\paragraph{Fubini's Theorem - Discontinuous}
Let \(f: R \to \mathbb{R}\) be bounded on a rectangular domain
\(R = [a, b] \times [c, d]\) with the discontinuities of \(f\) confined to
a finite union of graphs of continuous functions.
If \(\int_c^d\) exists for each \(x\int [a, b]\) then
\[\int\int_R = \int_a^b \left(\int_x^d f(x, y) dy\right) dx.\]
Since \(f\) is not continuous then there is no guarantee that these integrals exist.


\subsection{Uniform Continuity, Leibniz}

\paragraph{Uniform Continuity}
The function \(f: \Omega\subset \mathbb{R}^n \to \mathbb{R}^m\)
is uniformly continuous on \(\Omega\) if for all \(x, y\in\Omega\) and
for all \(\epsilon > 0\), there exists \(\delta\) such that
\[
    d(x, y) < \delta \quad \Rightarrow \quad d(f(x), f(y)) < \epsilon.
\]
\(\delta\) may depend on \(x\) but, given an \(\epsilon\),
the same \(\delta\) must work for all \(x\).

\paragraph{Continuity and Uniform Continuance}
A continuous function on a compact \(\Omega\) is uniformly continuous on \(\Omega\).

\paragraph{Leibniz' Rule}
Consider a function \(f: \mathbb{R}^2 \to \mathbb{R}\) that is continuous
on \(\Omega\) and \(\frac{\partial f}{\partial x}\) is uniformly continuous on \(\Omega\).

If,
\[F(x) = \int_a^b f(x, y)\]
then
\[F'(x) = \frac{d}{dx} \int_a^b f(x, y)dy = \int_a^b \frac{\partial f}{\partial x}(x,y) dy.\]

\subsection{Alternate Coordinates}
\paragraph{Change of Variable}
Suppose that \(F: \Omega\in\mathbb{R}^n \to \mathbb{R}^n\) is \(C^1\),
\(\det(J_xF \neq 0)\) for \(x\in\Omega\) and \(F\) is one-to-one.
Then, if \(f\) is integrable on \(\Omega ' = F(\Omega)\)
\[
    \int_{\Omega '} f(x, y)
    =
    \int_{\Omega} (f\circ F) |\det J F|.
\]

As an alternate notation consider,
\[
    \int_{\Omega '} f(x, y)dx dy
    =
    \int_\Omega f( x(u, v), y(u, v) ) |\det JF| du dv.
\]
where
\[
    \det Jf = \det \begin{pmatrix}
        \frac{\partial x}{\partial u} & \frac{\partial x}{\partial v}
        \frac{\partial y}{\partial u} & \frac{\partial y}{\partial v}
    \end{pmatrix}.
\]

Note that \(F\) is the function which maps the change of variable.

\subsection{Mass, Centre of Mass, Centroid}
For the following section, suppose that \(\Omega\subset \mathbb{R}\)
with a density function \(p: \Omega to \mathbb{R}\).

Note that this can be generalised to \(n\) dimensions by the \(n\)-th integral
rather than a double integral

\paragraph{Mass}
The total mass is
\[M = \int_\Omega p(x, y) dx dy.\]

\paragraph{Center of Mass}
The coordinates for centre of mass follow as such:
\begin{itemize}
    \item \(\bar{x} = \int\int_\Omega x p(x, y) dx dy\)
    \item \(\bar{y} = \int\int_\Omega y p(x, y) dx dy\)
\end{itemize}



