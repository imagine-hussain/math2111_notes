\section{Path Integrals}

\subsection{Path Integrals}


\paragraph{Path (Scalar Line) Integrals}
Suppose that a vector-valued function \(c(t)\) parametrises a 
curve \(C\) for \(t\in [a, b]\).
The scalar line integral may be thought as the integral of along \(c\).

\paragraph{Computing a Scalar Line Integral}
Let \(c(t)\) parametrise a curve \(C\) for \(t\in [a, b]\).
Assume that \(f(x, y, z)\) and \(c(t)\) are continuous.
Then,
\[
    \int_{C} f(x, y, z) ds
    =
    \int_{a}^{b} f\left(c\left(t\right)\right) \cdot ||c(t)|| dt.
\]

\paragraph{Elementary Properties of Path Integrals}
\begin{itemize}
    \item \(\int_C f_1 ds + \int_C f_2 ds = \int_C(f_1+f_2)ds\),
    \item \(\int_C\lambda f ds = \lambda \int_C f ds, \quad
    \lambda\in \mathbb{R}\).
\end{itemize}

\subsection{Applications Of Path Integrals}
Suppose that \(\delta = \delta(x, y, z)\) which is a density function.

\paragraph{Mass}
\[M = \int_C \delta(x, y, z) ds.\]

\paragraph{First Moments About the Coordinate Plane}
\begin{itemize}
    \item \(M_{yz} = \int_C x \delta ds\)
    \item \(M_{xz} = \int_C y \delta ds\)
    \item \(M_{xy} = \int_C z \delta ds\)
\end{itemize}

\paragraph{Coordinates of Center of Mass}  
\begin{itemize}
    \item \(\bar{x} = \frac{M_{yz}}{M}\)
    \item \(\bar{y} = \frac{M_{xz}}{M}\)
    \item \(\bar{z} = \frac{M_{xy}}{M}\)
\end{itemize}

\paragraph{Moments of Inertia About Axes}
\begin{itemize}
    \item \(I_x = \int_C (y^2 + z^2) \delta ds\)
    \item \(I_x = \int_C (x^2 + z^2) \delta ds\)
    \item \(I_x = \int_C (x^2 + y^2) \delta ds\)
\end{itemize}


\section{Vector Line Integrals}

\subsection{Vector Line Integrals}


\paragraph{Vector Line Integrals}  
Vector line integrals are different from scalar line integrals in the
sense that to define a vector line integral, we must specify a direction
along the path or curve \(C\).

\paragraph{Computing a Vector Line Integral}
Let \(c(t)\) parametrise an oriented curve \(C\) for \(t\in [a, b]\).
Then, 
\[
    \int_C F\cdot ds = \int_a^b F(c(t)) \cdot c'(t).
\]

\paragraph{Link to Path Integrals}
Suppose that \(C\) is a smooth curve with a parametrisation \(c(t)\)
for \(t\in[a, b]\) where \(c(t)\) is continuously differentiable and
\(c'(t) \neq 0\) for all \(t\in [a, b]\).

Then, \(c'(t)\) is a non-zero tangent vector pointing in the forward direction and the unit tangent vector is 
\[T(c(t)) = \frac{c'(t)}{||c'(t)||}.\]
Then,
\[
    \int_C F\cdot ds = \int_C F \cdot T ds.
\]

\paragraph{Summing Paths}
Suppose that \(C\) is made of \(n\) finitely many paths \(C_i\).
Then, \(C = \sum_{i}^n C_i\).
Note that all the curves must be joined end to end.
Then,
\[\int_C F \cdot ds = \sum_{i}^n \int_{C_i} F \cdot ds.\]

\paragraph{Work and Other Alternative Notations}
Suppose that \(c(t) = (x(t), y(t), z(t))\) and
\(F = (M, N, P)\).
Then, we denote work as any of the following notations
\begin{align*}
    W
    &= \int_C F\cdot ds \\
    &= \int_a^b \left(
        M \frac{dx}{dt} + N \frac{dy}{dt} + P \frac{dz}{dt}
    \right) dt \\
    &= \int_C M ds + N ds + P ds.
\end{align*}

\paragraph{Properties fo Line Integrals}
% TODO - 
\begin{itemize}
    \item Linearity
    \item Reversing Orient
\item Additivity
\end{itemize}

\paragraph{Flow Integrals and Circulation}
Suppose that \(F\) represents a velocity field of a fluid flowing
through a region in space. Then, the flow across a curve may be
defined as the following
\[
\text{Flow} = \int_a^b F\cdot \hat{T}ds.
\]
This integral is called the flow integral. If the curve is a closed
loop then this is called the \textit{circulation} around the curve.

\paragraph{Flux in the Plane}
If \(C\) is a smooth closed curve in the domain of a continuous vector field
\(F = M(x, y)i + N(x, y)j + N(x, y)j\)
and, \(n\) is the outward pointing unit-normal on \(C\) then, the flux
of \(f\) across \(C\) is the following expression
\[\int_C F\cdot \hat{n} ds.\]

\paragraph{Calculating Flux Across a Smooth Closed Plane Curve}
Suppose that \(F = Mi = Nj\). Let \(G = -N, M\)
\[
\text{Flux of \(F\) across \(C\)} = 
\oint_C M dy - N dx = \oint_C G ds
\]

\subsection{Fundamental Theorem of Line Integrals}

\paragraph{Gradient Fields}
A vector field \(F\) is called a gradient vector field if there
exists a real-valued function \(\phi\) such that
\(F = \nabla \phi\). 
That is,
\[
    \begin{pmatrix} M \\ N \\ P \end{pmatrix}
    =
    \begin{pmatrix}
        \dfrac{\partial \phi}{\partial x} \\
        \dfrac{\partial \phi}{\partial y} \\
        \dfrac{\partial \phi}{\partial z}
    \end{pmatrix}.
\]
If such a function \(\phi\) exists then \(\phi\) is called the potential function of \(F\)
where, \(F\) is \textit{conservative}.

\paragraph{Fundamental Theorem for Gradient Vector Fields}
If \(F = \nabla \phi\) on a domain \(\mathcal{D}\). Then,
for all oriented curves \(C\) in \(\mathcal{D}\) with an initial
point \(P\) and a terminal point \(Q\),
\[
    \int_C F\cdot ds = \phi(Q) - \phi(P).
\]
The integral is independent of the path.

\paragraph{Cross Partials of Gradient Vector Fields are Equal}
Let \(F = (F_1, F_2, F_3)\) be a gradient vector field whose
components have continuous partial derivatives.
Then, the cross partials are equal. That is,
\begin{align*}
    \frac{\partial F_1}{\partial y} &= \frac{\partial F_2}{\partial x},\\
    \frac{\partial F_2}{\partial z} &= \frac{\partial F_3}{\partial y},\\
    \frac{\partial F_3}{\partial x} &= \frac{\partial F_1}{\partial z}.
\end{align*}
Equivalently,
\[\curl F = 0.\]

\subsection{Green's Theorem}
Green's Theorem connects double integrals and line integrals.


\paragraph{Green's Theorem: Flux Divergence or Normal Form}
Let \(D\) be a bounded simple region in \(\mathbb{R}^2\) with a nonempty interior
whose boundary consists of a finite number of smooth curves.
Let \(C\)be the boundary of \(D\) with a positive (counter-clockwise) direction.
Let \(F=Mi+Nj\) be a continuously different boundary vector field on D.

Then, the outward flux of \(F\) across the curve \(C\) equals the double
integral of divergence \(\div F\) over \(D\). That is,
\[
    \oint_C (F\cdot \hat{n})ds = \oint_C -Ndx + Mdy
    = \int\int_D  \left(\frac{\partial M}{\partial x} + \frac{\partial N}{\partial Y}\right).
\]

Once again, note the assumptions:
\begin{itemize}
    \item \(D\) is bounded and simple with a non-empty interior,
    \item The boundary \(C\) is oriented in the positive (counter-clockwise) direction,
    and is the finite union of smooth curves,
    \item The vector field \(F\) is continuously differentiable on \(D\).
\end{itemize}

\paragraph{Green's Theorem: Circulation-Curl or Tangential Form}
Let \(D\) be a bounded simple region in \(\mathbb{R}^2\) with a nonempty interior
whose boundary consists of a finite number of smooth curves.
Let \(C\)be the boundary of \(D\) with a positive (counter-clockwise) direction.
Let \(F=Mi+Nj\) be a continuously different boundary vector field on D.

Then, the counter-clockwise circulation of \(F\) around \(C\) equals the double
integral of \((\curl F)\cdot k\) over \(D\).
That is,
\[
    \oint_C (F\cdot \hat{T}) ds =      
    \oint_C Mdx + Ndy = 
    \int\int_D  \left(\frac{\partial N}{\partial x} - \frac{\partial M}{\partial y}\right) dx dy.
\]

\paragraph{Area of a Region}
Let \(D\) be a simple and bounded region with a non-empty interior and let \(C\)
be the boundary of \(D\) which is the finite union of smooth curves.
Then, the area of \(D\) can be calculated as such
\[
\text{Area}(D) = \frac{1}{2} \oint_C \left(-y dx + xdy\right).
\]
