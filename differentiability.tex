\section{Differentiation}

\subsection{Differentiability, Derivatives and Affine Approximations}

\paragraph{Differentiability in \(\mathbb{R}\)}
A function \(f: \mathbb{R}\to \mathbb{R}\) being differentiable at some \(a\in \mathbb{R}\)
implies that there exists a \textit{good} straight-line approximation to \(f\) at \(a\) called a \textit{tangent line}.
This function may be found as
\[T(x) = f(a) + f'(a)(x-a) = f(a) -f'(a)a + f'(a)x = y_0 + L(x)\]
where for all \(a\), \(y= f(a) - f'(a)a\) and \(L: \mathbb{R}\to \mathbb{R} = f'(a)x\).

Recall that \[f'(a) = \lim_{x\to a} \frac{f(x) - f(a)}{x-a}\].

\paragraph{Affine Maps}
A function \(T: \mathbb{R}^n \to \mathbb{R}^m\) being affine means that there exists
a \(y_0\) such that for all \(x\in \mathbb{R}^n\)
\[T(x) = y_0 + L(x)\].

In \(T: \mathbb{R}\to \mathbb{R}\) this sis of the form \(y = mx+b\).

A function \(f: \mathbb{R}\to \mathbb{R}\) is differentiable if there is 
a good affine approximation to \(f\) of the form
\[T(x) = f(a) - f'(a)a + f'(a)x.\]
In this context good implies that \(f'(x)\) is defined in the usual manner
and exists.

\paragraph{Differentiability in \(\mathbb{R}^n\to \mathbb{R}^n\)}

A function \(f: \Omega\subset \mathbb{R}^n \to \mathbb{R}^m\) is differentiable for some
\(a\in\mathbb{\Omega}\) if there exists a linear map \(L: \mathbb{R}n\to \mathbb{R}^m\)
such that
\[
\lim_{x\to a} \frac{
    \left|\left|f(x) - f(a) -L(x-a)\right|\right|
} {
    \left|\left|L(x-a)\right|\right|
} = 0.
\]

Notation: the matrix of the linear map \(L\), the derivative of \(f\) at
\(a\) is denoted by \(D_af\).

\paragraph{Delta Epsilon Definition of Differentiability}
A function \(f: \Omega\subset \mathbb{R}\to \mathbb{R}^m\) is 
differentiable on \(a\in \Omega\) if there is a linear map \(L: \mathbb{R}^n\to \mathbb{R}^m\)
such that \(\forall \epsilon > 0 \exists \delta > 0 \)
such that for all \(x\in \Omega\)
\[
\left|\left|x - a\right|\right| < \delta
\Rightarrow
\left|\left|f(x) - f(a) - L(x-a)\right|\right|
< \epsilon\left|\left|x - a\right|\right|. 
\]

\paragraph{Clairaut's Theorem / Mixed Derivative Theorem}
Suppose \(
f, \frac{\partial f}{\partial x_i}, \frac{\partial f}{\partial x_j},
    \frac{\partial^2 f}{\partial x_i \partial x_j},
    \frac{\partial^2 f}{\partial x_j \partial x_i}
\)
all exist and are continuous on an open set around \(a\) then
\[
    \frac{\partial^2 f}{\partial x_i \partial x_j}
    =
    \frac{\partial^2 f}{\partial x_j \partial x_i}.
\]
That is, the partial derivatives commute.

\paragraph{Differentiability and Continuity} Differentiability implies continuity.
However, continuity does not imply differentiability.

\subsection{Gradients, Affine Approximations and Matrices}

\paragraph{Jacobian Matrices}



% 
% - Jacobian Matrix
% - Differentiable -> continuous
% - Gradient
% - Affine Approximations
% - Tangent Planes
% - Chain rule
% Directional Derivative
% Tangent Planes
% Tangent Lines
% Taylor Series
% Taylor Theorem
% Max, Min, Saddle Points
% Sylvester's Criterion
% Lagrange Multiplies
% Inverse Theorem
% Implicit Function Theorem