\section{Differentiation}

\subsection{Differentiability, Derivatives and Affine Approximations}

\paragraph{Differentiability in \(\mathbb{R}\)}
A function \(f: \mathbb{R}\to \mathbb{R}\) being differentiable at some \(a\in \mathbb{R}\)
implies that there exists a \textit{good} straight-line approximation to \(f\) at \(a\) called a \textit{tangent line}.
This function may be found as
\[T(x) = f(a) + f'(a)(x-a) = f(a) -f'(a)a + f'(a)x = y_0 + L(x)\]
where for all \(a\), \(y= f(a) - f'(a)a\) and \(L: \mathbb{R}\to \mathbb{R} = f'(a)x\).

Recall that \[f'(a) = \lim_{x\to a} \frac{f(x) - f(a)}{x-a}\].

\paragraph{Affine Maps}
A function \(T: \mathbb{R}^n \to \mathbb{R}^m\) being affine means that there exists
a \(y_0\) such that for all \(x\in \mathbb{R}^n\)
\[T(x) = y_0 + L(x)\].

In \(T: \mathbb{R}\to \mathbb{R}\) this sis of the form \(y = mx+b\).

A function \(f: \mathbb{R}\to \mathbb{R}\) is differentiable if there is 
a good affine approximation to \(f\) of the form
\[T(x) = f(a) - f'(a)a + f'(a)x.\]
In this context good implies that \(f'(x)\) is defined in the usual manner
and exists.

\paragraph{Differentiability in \(\mathbb{R}^n\to \mathbb{R}^n\)}

A function \(f: \Omega\subset \mathbb{R}^n \to \mathbb{R}^m\) is differentiable for some
\(a\in\mathbb{\Omega}\) if there exists a linear map \(L: \mathbb{R}n\to \mathbb{R}^m\)
such that
\[
\lim_{x\to a} \frac{
    \left|\left|f(x) - f(a) -L(x-a)\right|\right|
} {
    \left|\left|L(x-a)\right|\right|
} = 0.
\]

Notation: the matrix of the linear map \(L\), the derivative of \(f\) at
\(a\) is denoted by \(D_af\).

\paragraph{Delta Epsilon Definition of Differentiability}
A function \(f: \Omega\subset \mathbb{R}\to \mathbb{R}^m\) is 
differentiable on \(a\in \Omega\) if there is a linear map \(L: \mathbb{R}^n\to \mathbb{R}^m\)
such that \(\forall \epsilon > 0 \exists \delta > 0 \)
such that for all \(x\in \Omega\)
\[
\left|\left|x - a\right|\right| < \delta
\Rightarrow
\left|\left|f(x) - f(a) - L(x-a)\right|\right|
< \epsilon\left|\left|x - a\right|\right|. 
\]

\paragraph{Clairaut's Theorem / Mixed Derivative Theorem}
Suppose \(
f, \frac{\partial f}{\partial x_i}, \frac{\partial f}{\partial x_j},
    \frac{\partial^2 f}{\partial x_i \partial x_j},
    \frac{\partial^2 f}{\partial x_j \partial x_i}
\)
all exist and are continuous on an open set around \(a\) then
\[
    \frac{\partial^2 f}{\partial x_i \partial x_j}
    =
    \frac{\partial^2 f}{\partial x_j \partial x_i}.
\]
That is, the partial derivatives commute.

\paragraph{Differentiability and Continuity} Differentiability implies continuity.
However, continuity does not imply differentiability.
The proof of this is contingent on the fact that for \(x\in \mathbb{R}^n\)
and a \(m\times n\) matrix \(L\)
\[\lim x\to 0 \left|\left|Lx\right|\right| = 0.\]

\paragraph{Partial Derivatives and Differentiability}
Suppose that \(\Omega\subset \mathbb{R}^n\) is open and \(f: \Omega \to \mathbb{R}^m\).
If all partial derivatives \(\dfrac{\partial f_j}{\partial x_i}\) exist
for integers \(i \in [1, n]\), \(j\in [1, m]\) then
\(f\) is differentiable on \(\Omega\).

\subsection{Gradients, Affine Approximations and Matrices}

\paragraph{Jacobian Matrices}
Suppose that all partial derivatives of \(f:\Omega \subset \mathbb{R}^n \to \mathbb{R}^m\)
exist for some \(a\in \Omega\). Then, the Jacobian matrix of \(f\)
\[
    J_a f = 
    \begin{pmatrix}
    \dfrac{\partial f_1}{\partial x_1} & \dfrac{\partial f_1}{\partial x_2} \cdots & \dfrac{\partial f_1}{\partial x_n} \\
    \dfrac{\partial f_2}{\partial x_1} & \dfrac{\partial f_2}{\partial x_2} \cdots & \dfrac{\partial f_2}{\partial x_n} \\
    \vdots & \vdots & \ddots \vdots \\
    \dfrac{\partial f_n}{\partial x_1} & \dfrac{\partial f_n}{\partial x_2} \cdots & \dfrac{\partial f_n}{\partial x_n} \\
    \end{pmatrix}
\]
may be evaluated at a point \(a\).
Where \(f\) is differentiable, its derivative is given by the Jacobian matrix.

Note however, that the Jacobian Matrix may exist even where \(f\) is not differentiable.

\subsection{Gradients, Tangent Planes and Affine Approximations}

\paragraph{Gradient}
For \(f: \Omega\subset \mathbb{R}^n\to \mathbb{R}\), if the Jacobian exists,
then it is given by the \(1\times n\) matrix
\[
    Jf = \begin{pmatrix}
        \dfrac{\partial f}{\partial x_1} \\
        \dfrac{\partial f}{\partial x_2} \\
        \cdots                           \\
        \dfrac{\partial f}{\partial x_n} \\
    \end{pmatrix}.
\]
This is equivalent to the gradient of \(f\). That is,
\[
    \text{grad}(f) = \grad f = 
    \begin{pmatrix}
        \dfrac{\partial f}{\partial x_1} \\
        \dfrac{\partial f}{\partial x_2} \\
        \cdots                           \\
        \dfrac{\partial f}{\partial x_n} \\
    \end{pmatrix}.
\]

\paragraph{Affine Approximations}
Allow \(f:\Omega\subset\mathbb{R}^n \to \mathbb{R}\) to be a differentiable
function at \(a\in \Omega\).
The best affine approximation to \(f\) at \(a\) may be written in terms of
the gradient vector as
\[
    T(x) = f(a) + \grad f(a) \cdot (x-a).
\]

\paragraph{Tangent Planes}
The tangent plane to a function \(z = f(x, y)\) is given by
\[ z = T(x, y).\]

\subsection{Chain Rule, Directional Derivatives and Tangent Planes}

\paragraph{Chain Rule}
Suppose that \(f: \Omega \subset \mathbb{R}^n \to \mathbb{R}^m\) and
\(g: \Omega' \subset \mathbb{R}^m \to \mathbb{R}^P\) where
\(f(\Omega) = \Omega'\).
If \(f\) and \(g\) are both differentiable then, so is
\(g\circ f: \Omega \to \mathbb{R}^p\) such that
\[
    D_a (g\circ f) = D_(f_(a))g D_a f.
\]
Equivalently,
\[
    D(g\circ f)(a) = Dg(f(a)) Df(a).
\]

\paragraph{Directional Derivative}
The directional derivative of 
\(f: \Omega \subset \mathbb{R}^n \to \mathbb{R}^m\)
in the direction of the unit vector \(u\) at a point \(a\in\Omega\) is
\[
    D_u f(a) = f_{u}'(a)
    =
    \lim_{t\to 0}  \dfrac{f(a + t u) - f(a)}{t}.
\]

Equivalently, if \(f: \Omega \subset \mathbb{R}^n \to \mathbb{R}\)
is differentiable at \(a\) then for a unit vector \(u\)
\[
    D_u f(a) = Df(a) \cdot u = \grad f(a)\cdot u.
\]

Alternatively, allowing \(\theta\) to be the angle between
\(\grad f(a)\) and \(u\),
\[D_u f(a) = |\grad f(a)| \cdot |u| \cdot \cos\theta.\]

\paragraph{Tangent Planes}

Consider the surface in \(\mathbb{R}^3\) defined by \(\phi (x, y, z) = \lambda\).
where \(\lambda\) is constant and \(\phi\) is differentiable.

Let \(c(t) = \left(c_1(t), c_2(t), c_3(t)\right)\) be a differentiable curve
lying on the vector space with a tangent vector given by 
\(c'(t) = \left(c'_1(t), c'_2(t), c'_3(t)\right)\).

Since all points \(c(t)\) lie on the surface, \(\phi(c(t)) = \lambda\).
Thus,
\[
    D(\phi (c(t))) Dc(t) = 0 \Rightarrow \grad \phi c'(t) = 0.
\]
Therefore, all curves passing through a point \(P\) on the surface have
tangent vector normal to \(\grad \phi\). Thus, they all lie in the tangent
plane at \(P\).

\subsection{Taylor Series and Theorem}

\paragraph{Taylor's Theorem}
For all continuous and differentiable functions \(f: \mathbb{R}\to \mathbb{R}\),
\[
    f(x)
    \approx
    P_k(a) = 
    \sum_{n=0}^{k} \frac{1}{n!}f^{(n)}(x) (x-a)^n.
    + R
\]
where the remainder \(R\) is
\[
    R = \frac{1}{(k+1)!} f^{(k+1)}(z) (x-a)^k.
\]
for some \(z\) between \(x\) and \(a\).

\(P_0, P_1,  P_2, P_3\) are the best constant, affine, quadratic, cubic approximations.

\paragraph{Generalizing Taylor's Theorem}

% - Tangent Planes
% - Chain rule
% Directional Derivative
% Tangent Planes
% Tangent Lines
% Taylor Series
% Taylor Theorem
% Max, Min, Saddle Points
% Sylvester's Criterion
% Lagrange Multiplies
% Inverse Theorem
% Implicit Function Theorem