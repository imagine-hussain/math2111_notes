
\section{Analysis}

\subsection{Assumed}
\paragraph{Assumed Concepts from Real Single-Variable Calculus}
\begin{itemize}
    \item limits
    \item continuity
    \item differentiability
    \item integrability
\end{itemize}

\paragraph{Assumed Theorems}
\begin{itemize}
    \item Min/ Max Theorem
    \item Intermediate Value Theorem
    \item Mean Value Theorem
\end{itemize}

\subsection{Limits}
Recall that \(\lim_{x\to a} f(x) = L\) requires that for all
\(\epsilon > 0\), there exists a \(\delta > 0\) such that
if \(|x-a| < \delta\)
then
\[|f(x) - L|  < \delta.\]



\subsection{Metrics}
We have metrics (distance functions) as
\[m: \mathbb{R}^n\times\mathbb{R}^n \to \mathbb{R}\]
satisfying the following 3 axioms.
\begin{itemize}
    \item \textbf{Positive Definite} such that for all \(x,y\in\mathbb{R}^n\),
    \(m(x,y) > 0\) and, \(m(x,y) = 0\ \Leftrightarrow x = y\).
    \item \textbf{Symmetric} \(m(x,y) = m(y, x)\).z
    \item \textbf{Triangle Inequality} such that for all \(x,y,z\in\mathbb{R}^n\),
    \(m(x,y) + m(y, z) \leq m(x, z)\).
\end{itemize}

\paragraph{Euclidian Distance}
We allow the Euclidian distance to be defined as
\[d_n(x,y) := ||x-y|| = \sqrt{\sum_{i=1}^{n} (x_i - y_y)^2}\]
We often allow \(d\) to be \(d_2\).

\paragraph{Norms} Norms will be revisited in the Fourier Series section. They can be thought of as the length of an element in vectors space.

\paragraph{Equivalent Metrics}
Two metrics \(d\) and \(\delta\) are considered equal if 
there exists constants \(0 < c < C < \infty\) such that
\[ c\delta (x,y) \leq d(x,y) \leq C \delta(x, y).\]

\subsection{Limits of Sequences}

\paragraph{Balls} A ball around \(\vec{a}\in \mathbb{R} \) is of radius \( \epsilon \) is the set
\[B(\vec{a}, \epsilon), = \{x\in \mathbb{R} : d(\vec{a}, x) < \epsilon\}.\]

\paragraph{Limit in Sequence} 
For a sequence \( \{x_i\} \) of points in \( \mathbb{R}^n \),
\(x\) is the limit of the sequence if and only if 
\[\forall \epsilon > 0 \exists N \text{ such that } n \geq N \implies d(x, x_n) \leq \epsilon. \]
Equivalently,
\[\forall \epsilon > 0 \exists N \text{ such that } n \geq N \implies d(x, x_n) \in B(x, \epsilon).\]

\paragraph{Theorems with Limits of Sequences}
\begin{align*}
    \text{A sequence \(x_k\) converges to a limit \(x\)}\\
    &\Leftrightarrow \text{The components of \(x_k\)} \\
    & \quad \text{converge to the componetents of \(x\)} \\
    & \Leftrightarrow d(x_k, x) \rightarrow 0.
\end{align*}

\paragraph{Limits and Equivalent Metrics}
Suppose that \(d\) and \(\delta\) are two equivalent metrics.
That is, \(cd(x,y) \leq \delta (x, y) \leq Cd(x,y)\) for \(c, C > 0\).

Considering \(d\) as the metric, suppose that 
\[x_k \to x\quad\text{ for } x_k, x \in\mathbb{R}^n.\]
That is,
\[\forall \epsilon > 0, \exists K : k \geq K \implies d(x_k, x) < \epsilon.\]
Using \(\delta\), we may make an equivalent statement, choosing
\(\epsilon > 0\) such that \(\epsilon' = C\epsilon\).
Considering that \(e > 0 \implies \exists K : \forall k \geq K
\implies d(x_k, x) < \epsilon\) then, 
\[\delta (x_k, x) \leq Cd (x_k, x) < C\epsilon = \epsilon'.\]
That is, \(\delta (x_k, x) < \epsilon'.\) Hence \(x_k \to x\)
using an equivalent metric \(\delta\).

\paragraph{Cauchy Sequences}
A sequence \(\{x_K\}\in\mathbb{R}\) is a Cauchy sequence if
\[
    \exists \epsilon > 0 \text{ such that } k,I > K 
    \implies d(x_k, x_l) < \epsilon.
\]

\paragraph{Cauchy Sequences and Convergence}
The following are equivalent:
\[
    \text{A sequence \(\{x_k\}\) converges in \(\mathbb{R}^2\)}
    \quad \Longleftrightarrow \quad
    \text{\(\{x_k\}\) is a Cauchy Sequence}.
\]

\subsection{Open and Closed Sets}

\paragraph{Definitions}
Consider \(x_k \)
\begin{itemize}
    \item \(x_0\in\Omega\) is an \underline{interior points} of \(\Omega\) if there is a ball around \(x\) completely contained in \(\Omega\). That is, there exists a \(\epsilon > 0\) such that \(B(x_0, \epsilon) \subseteq \Omega\).
    \item  \(\Omega\) is open if every point of \(\Omega\) is an interior point.
    \item \(\Omega\) is closed if its complement is open.
    \item \(x_0\in\Omega\) is a \underline{boundary point} of \(\Omega\) if every ball around \(x_0\) contains points in \(\Omega\) and points not in \(\Omega\).
\end{itemize}

\paragraph{Closed Sets}
A set \(\Omega \subset \mathbb{R}\) is closed iff and only if it contains all of its boundary points.

\paragraph{Limit Points and Sets}
\(x_0\) is a limit point of \(\Omega\) if there is a sequence \(\{x_i\}\) 
in \(\Omega\) with limit \(x_0\) and \(x_i \neq x\).

\begin{itemize}
    \item Every interior points of \(\Omega\) is a limit point of \(\Omega\).
    \item \(x_0\) is not necessarily in \(Omega\)
    \item A set is closed \(\Leftrightarrow\) it contains all of its limit points.
\end{itemize}

\paragraph{Variations of a Set}
Consider the set \(\Omega \in \mathbb{R}^n\).
\begin{itemize}
    \item The \underline{interior} of \(\Omega\) is the set of all its interior points.
    \item The \underline{boundary} \(\partial \Omega)\) of \(\Omega\) is the set of all its boundary points.
    \item The \underline{closure} of \(\Omega\): \(\bar{\Omega} = \Omega \cup \partial \Omega\).
\end{itemize}
The interior is the largest open subset and the closure is the smallest closed set containing \(\Omega\).

\paragraph{Limit of a Function at a Point}
For \(f: \Omega \subset \mathbb{R}^n \to \mathbb{R}^m\), 
\(\lim_{x\to x_0}\) means that 
\begin{align*}
    \forall \epsilon \exists \delta > 0\text{ such that for } x\in\Omega: \\
    0 < d(x, x_0) < \delta \implies d(f(x), b) < \epsilon.
\end{align*}
Alternatively,
\[
    x\in B(x_0, \delta) \backslash \{x_0\}
    \implies f(x)\in B(b, \epsilon).
\]
It is sufficient to consider the limits of the components of a function.

\paragraph{Limits and sequences}
The limit \(\lim_{x\to a} f(x) = b\) exists if and only if,
\(\lim_{k\to\infty} f(x_k) = b\)  for all sequences \({x_k}\) such that 
\(x_k\) is an element of \(\Omega\) and,
\(\lim_{k\to\infty} x_k = a\).

This is very helpful for showing that a limit does not exists.

\subsection{Pinching and IVT Theorem}
\paragraph{Pinching Theorem}
\paragraph{IVT}
see 1141
