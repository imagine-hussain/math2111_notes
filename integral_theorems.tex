\section{Integral Theorems}

\subsection{Stokes' Theorem}
\paragraph{Stokes' Theorem}
With Stokes' Theorem, we can compute line integrals over closed curves by
using surface integrals.
Suppose that
\begin{itemize}
    \item \(S\) is a smooth oriented surface,
    \item \(\partial S\) is the boundary of \(S\) in the anti-clockwise
    direction when looking at \(S\) from the positive direction,
    \item \(F\) is a continuously differentiable vector field on \(S\).
\end{itemize}

Then, \[
    \iint_S (\curl F) \cdot ds = \oint_{\partial S} F \cdot ds.
\]

\subsection{Gauss' Divergence Theorem}
The divergence theorem allows for computation of surface integrals
over close surfaces with a triple integral over the 3D solid bound by
the surface.
If 
\begin{itemize}
    \item The region \(W\subset \mathbb{R}^3\) is a bounded, solid, simple region.
    \item \(S\) is the piecewise smooth boundary, oriented so that the normal
    vector points outwards
    \item \(F\) is a \(C^1\) vector field on \(W\).
\end{itemize}
Then, \[
    \iint_S F\cdot dS = \iiint_W \div F \cdot dV.
\]
This is very helpful for complex vector fields with simpler expressions for
divergence.

Note that \(\iiint_W dV\) is the volume of \(W\).
Also note that for a conversion from \(x, y, z \to r, \theta, \phi\),
the Jacobian determinant is \(r^2 \sin \phi\).

