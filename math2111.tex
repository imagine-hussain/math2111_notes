\documentclass[12pt, letterpaper]{article}

\usepackage[margin=1in]{geometry}
\usepackage{amsmath}
\usepackage{amssymb}
\usepackage{mathtools}
\usepackage{fancyhdr}
\usepackage[utf8]{inputenc}
\usepackage{dirtytalk}                      % \say command for quotes
\usepackage{ wasysym }
\usepackage{graphicx}

\usepackage{hyperref}
\hypersetup {                                % Formatting for hyperlinks
    colorlinks,
    citecolor=black,
    filecolor=black,
    linkcolor=black,
    urlcolor=black
}

\graphicspath{'./assets/'}								% Path for images

\pagestyle{fancy}
\setlength{\headheight}{15pt}
\renewcommand{\headrulewidth}{0pt}
\renewcommand{\footrulewidth}{0pt}	
\title{Higher Several Variable Calculus \\ Math2111 UNSW}
\author{Hussain Nawaz \\ hussain.nwz000@gmail.com}
\date{2022T1}

\rhead{}
\lhead{}


\begin{document}
\maketitle
\tableofcontents
\newpage


    \section{Curves and Surfaces}
    \subsection{Curves}
    \paragraph{Curves}
    A curve in \(\mathbb{R}^n\) is a vector function
    \[\mathbf{c}: I \to\mathbb{R}^n,\]
    where \(I\) is an interval in \(\mathbb{R}\).

    \paragraph{Forms / Notations}
    Curves may be defined in the following ways:
    \begin{itemize}
        \item \textbf{Parametrically} by \(c(t) = \left(x_1(t), x_2(t), \dots, x_n(t)\right)\)
        \item \textbf{Cartesian} by eliminating the \(t\) variable to get \(y\) in terms of \(x\)
        \item \textbf{Implicitly} As \(F(x, y) = 0\).
    \end{itemize}

    % 
    % 
    % 
    % 
    % 
    % 
    % 
    % 
    % 
    \newpage

    \section{Analysis}

    \subsection{Assumed}
    \paragraph{Assumed Concepts from Real Single-Variable Calculus}
    \begin{itemize}
        \item limits
        \item continuity
        \item differentiability
        \item integrability
    \end{itemize}

    \paragraph{Assumed Theorems}
    \begin{itemize}
        \item Min/ Max Theorem
        \item Intermediate Value Theorem
        \item Mean Value Theorem
    \end{itemize}
    
    \subsection{Limits}
    Recall that \(\lim_{x\to a} f(x) = L\) requires that for all
    \(\epsilon > 0\), there exists a \(\delta > 0\) such that
    if \(|x-a| < \delta\)
    then
    \[|f(x) - L|  < \delta.\]
    
    
    
    \subsection{Metrics}
    We have metrics (distance functions) as
    \[m: \mathbb{R}^n\times\mathbb{R}^n \to \mathbb{R}\]
    satisfying the following 3 axioms.
    \begin{itemize}
        \item \textbf{Positive Definite} such that for all \(x,y\in\mathbb{R}^n\),
        \(m(x,y) > 0\) and, \(m(x,y) = 0\ \Leftrightarrow x = y\).
        \item \textbf{Symmetric} \(m(x,y) = m(y, x)\).z
        \item \textbf{Triangle Inequality} such that for all \(x,y,z\in\mathbb{R}^n\),
        \(m(x,y) + m(y, z) \leq m(x, z)\).
    \end{itemize}

    \paragraph{Euclidian Distance}
    We allow the Euclidian distance to be defined as
    \[d_n(x,y) := ||x-y|| = \sqrt{\sum_{i=1}^{n} (x_i - y_y)^2}\]
    We often allow \(d\) to be \(d_2\).

    \paragraph{Norms} Norms will be revisited in the Fourier Series section. They can be thought of as the length of an element in vectors space.

    \paragraph{Equivalent Metrics}
    Two metrics \(d\) and \(\delta\) are considered equal if 
    there exists constants \(0 < c < C < \infty\) such that
    \[ c\delta (x,y) \leq d(x,y) \leq C \delta(x, y).\]
    
    \subsection{Limits of Sequences}

    \paragraph{Balls} A ball around \(\vec{a}\in \mathbb{R} \) is of radius \( \epsilon \) is the set
    \[B(\vec{a}, \epsilon), = \{x\in \mathbb{R} : d(\vec{a}, x) < \epsilon\}.\]
    
    \paragraph{Limit in Sequence} 
    For a sequence \( \{x_i\} \) of points in \( \mathbb{R}^n \),
    \(x\) is the limit of the sequence if and only if 
    \[\forall \epsilon > 0 \exists N \text{ such that } n \geq N \implies d(x, x_n) \leq \epsilon. \]
    Equivalently,
    \[\forall \epsilon > 0 \exists N \text{ such that } n \geq N \implies d(x, x_n) \in B(x, \epsilon).\]

    \paragraph{Theorems with Limits of Sequences}
    \begin{align*}
        \text{A sequence \(x_k\) converges to a limit \(x\)}\\
        &\Leftrightarrow \text{The components of \(x_k\)} \\
        & \quad \text{converge to the componetents of \(x\)} \\
        & \Leftrightarrow d(x_k, x) \rightarrow 0.
    \end{align*}

    \paragraph{Limits and Equivalent Metrics}
    Suppose that \(d\) and \(\delta\) are two equivalent metrics.
    That is, \(cd(x,y) \leq \delta (x, y) \leq Cd(x,y)\) for \(c, C > 0\).

    Considering \(d\) as the metric, suppose that 
    \[x_k \to x\quad\text{ for } x_k, x \in\mathbb{R}^n.\]
    That is,
    \[\forall \epsilon > 0, \exists K : k \geq K \implies d(x_k, x) < \epsilon.\]
    Using \(\delta\), we may make an equivalent statement, choosing
    \(\epsilon > 0\) such that \(\epsilon' = C\epsilon\).
    Considering that \(e > 0 \implies \exists K : \forall k \geq K
    \implies d(x_k, x) < \epsilon\) then, 
    \[\delta (x_k, x) \leq Cd (x_k, x) < C\epsilon = \epsilon'.\]
    That is, \(\delta (x_k, x) < \epsilon'.\) Hence \(x_k \to x\)
    using an equivalent metric \(\delta\).

    \paragraph{Cauchy Sequences}
    A sequence \(\{x_K\}\in\mathbb{R}\) is a Cauchy sequence if
    \[
        \exists \epsilon > 0 \text{ such that } k,I > K 
        \implies d(x_k, x_l) < \epsilon.
    \]

    \paragraph{Cauchy Sequences and Convergence}
    The following are equivalent:
    \[
        \text{A sequence \(\{x_k\}\) converges in \(\mathbb{R}^2\)}
        \quad \Longleftrightarrow \quad
        \text{\(\{x_k\}\) is a Cauchy Sequence}.
    \]
    

    
    \subsection{Soon}
    \(//\text{soon}^\text{tm}\)
\end{document}