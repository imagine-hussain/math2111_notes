\documentclass[12pt, letterpaper]{article}

\usepackage[margin=1in]{geometry}
\usepackage{amsmath}
\usepackage{amssymb}
\usepackage{mathtools}
\usepackage{fancyhdr}
\usepackage[utf8]{inputenc}
\usepackage{dirtytalk}                      % \say command for quotes
\usepackage{ wasysym }
\usepackage{graphicx}
\usepackage{physics}
\usepackage{esint}

% Dark Mode
% \usepackage{xcolor}
% \pagecolor[rgb]{0,0,0} %black
% \color[rgb]{1, 1, 1} %white


\usepackage{hyperref}
\hypersetup {                                % Formatting for hyperlinks
    colorlinks,
    citecolor=black,
    filecolor=black,
    linkcolor=black,
    urlcolor=black
}

\graphicspath{'./assets/'}								% Path for images

\pagestyle{fancy}
\setlength{\headheight}{15pt}
\renewcommand{\headrulewidth}{0pt}
\renewcommand{\footrulewidth}{0pt}	
\title{Higher Several Variable Calculus \\ Math2111 UNSW}
\author{Hussain Nawaz \\ hussain.nwz000@gmail.com}
\date{2022T1}

\rhead{}
\lhead{}


\begin{document}
\maketitle
\tableofcontents
\newpage


    \section{Curves and Surfaces}
    \subsection{Curves}
    \paragraph{Curves}
    A curve in \(\mathbb{R}^n\) is a vector function
    \[\mathbf{c}: I \to\mathbb{R}^n,\]
    where \(I\) is an interval in \(\mathbb{R}\).

    \paragraph{Forms / Notations}
    Curves may be defined in the following ways:
    \begin{itemize}
        \item \textbf{Parametrically} by \(c(t) = \left(x_1(t), x_2(t), \dots, x_n(t)\right)\)
        \item \textbf{Cartesian} by eliminating the \(t\) variable to get \(y\) in terms of \(x\)
        \item \textbf{Implicitly} As \(F(x, y) = 0\).
    \end{itemize}

    % 
    % 
    % 
    % 
    % 
    % 
    % 
    % 
    % 
    \newpage

    \section{Analysis}

    \subsection{Assumed}
    \paragraph{Assumed Concepts from Real Single-Variable Calculus}
    \begin{itemize}
        \item limits
        \item continuity
        \item differentiability
        \item integrability
    \end{itemize}

    \paragraph{Assumed Theorems}
    \begin{itemize}
        \item Min/ Max Theorem
        \item Intermediate Value Theorem
        \item Mean Value Theorem
    \end{itemize}
    
    \subsection{Limits}
    Recall that \(\lim_{x\to a} f(x) = L\) requires that for all
    \(\epsilon > 0\), there exists a \(\delta > 0\) such that
    if \(|x-a| < \delta\)
    then
    \[|f(x) - L|  < \delta.\]
    
    
    
    \subsection{Metrics}
    We have metrics (distance functions) as
    \[m: \mathbb{R}^n\times\mathbb{R}^n \to \mathbb{R}\]
    satisfying the following 3 axioms.
    \begin{itemize}
        \item \textbf{Positive Definite} such that for all \(x,y\in\mathbb{R}^n\),
        \(m(x,y) > 0\) and, \(m(x,y) = 0\ \Leftrightarrow x = y\).
        \item \textbf{Symmetric} \(m(x,y) = m(y, x)\).z
        \item \textbf{Triangle Inequality} such that for all \(x,y,z\in\mathbb{R}^n\),
        \(m(x,y) + m(y, z) \leq m(x, z)\).
    \end{itemize}

    \paragraph{Euclidian Distance}
    We allow the Euclidian distance to be defined as
    \[d_n(x,y) := ||x-y|| = \sqrt{\sum_{i=1}^{n} (x_i - y_y)^2}\]
    We often allow \(d\) to be \(d_2\).

    \paragraph{Norms} Norms will be revisited in the Fourier Series section. They can be thought of as the length of an element in vectors space.

    \paragraph{Equivalent Metrics}
    Two metrics \(d\) and \(\delta\) are considered equal if 
    there exists constants \(0 < c < C < \infty\) such that
    \[ c\delta (x,y) \leq d(x,y) \leq C \delta(x, y).\]
    
    \subsection{Limits of Sequences}

    \paragraph{Balls} A ball around \(\vec{a}\in \mathbb{R} \) is of radius \( \epsilon \) is the set
    \[B(\vec{a}, \epsilon), = \{x\in \mathbb{R} : d(\vec{a}, x) < \epsilon\}.\]
    
    \paragraph{Limit in Sequence} 
    For a sequence \( \{x_i\} \) of points in \( \mathbb{R}^n \),
    \(x\) is the limit of the sequence if and only if 
    \[\forall \epsilon > 0 \exists N \text{ such that } n \geq N \implies d(x, x_n) \leq \epsilon. \]
    Equivalently,
    \[\forall \epsilon > 0 \exists N \text{ such that } n \geq N \implies d(x, x_n) \in B(x, \epsilon).\]

    \paragraph{Theorems with Limits of Sequences}
    \begin{align*}
        \text{A sequence \(x_k\) converges to a limit \(x\)}\\
        &\Leftrightarrow \text{The components of \(x_k\)} \\
        & \quad \text{converge to the componetents of \(x\)} \\
        & \Leftrightarrow d(x_k, x) \rightarrow 0.
    \end{align*}

    \paragraph{Limits and Equivalent Metrics}
    Suppose that \(d\) and \(\delta\) are two equivalent metrics.
    That is, \(cd(x,y) \leq \delta (x, y) \leq Cd(x,y)\) for \(c, C > 0\).

    Considering \(d\) as the metric, suppose that 
    \[x_k \to x\quad\text{ for } x_k, x \in\mathbb{R}^n.\]
    That is,
    \[\forall \epsilon > 0, \exists K : k \geq K \implies d(x_k, x) < \epsilon.\]
    Using \(\delta\), we may make an equivalent statement, choosing
    \(\epsilon > 0\) such that \(\epsilon' = C\epsilon\).
    Considering that \(e > 0 \implies \exists K : \forall k \geq K
    \implies d(x_k, x) < \epsilon\) then, 
    \[\delta (x_k, x) \leq Cd (x_k, x) < C\epsilon = \epsilon'.\]
    That is, \(\delta (x_k, x) < \epsilon'.\) Hence \(x_k \to x\)
    using an equivalent metric \(\delta\).

    \paragraph{Cauchy Sequences}
    A sequence \(\{x_K\}\in\mathbb{R}\) is a Cauchy sequence if
    \[
        \exists \epsilon > 0 \text{ such that } k,I > K 
        \implies d(x_k, x_l) < \epsilon.
    \]

    \paragraph{Cauchy Sequences and Convergence}
    The following are equivalent:
    \[
        \text{A sequence \(\{x_k\}\) converges in \(\mathbb{R}^2\)}
        \quad \Longleftrightarrow \quad
        \text{\(\{x_k\}\) is a Cauchy Sequence}.
    \]
    
    \subsection{Open and Closed Sets}

    \paragraph{Definitions}
    Consider \(x_k \)
    \begin{itemize}
        \item \(x_0\in\Omega\) is an \underline{interior points} of \(\Omega\) if there is a ball around \(x\) completely contained in \(\Omega\). That is, there exists a \(\epsilon > 0\) such that \(B(x_0, \epsilon) \subseteq \Omega\).
        \item  \(\Omega\) is open if every point of \(\Omega\) is an interior point.
        \item \(\Omega\) is closed if its complement is open.
        \item \(x_0\in\Omega\) is a \underline{boundary point} of \(\Omega\) if every ball around \(x_0\) contains points in \(\Omega\) and points not in \(\Omega\).
    \end{itemize}

    \paragraph{Closed Sets}
    A set \(\Omega \subset \mathbb{R}\) is closed iff and only if it contains all of its boundary points.

    \paragraph{Limit Points and Sets}
    \(x_0\) is a limit point of \(\Omega\) if there is a sequence \(\{x_i\}\) 
    in \(\Omega\) with limit \(x_0\) and \(x_i \neq x\).

    \begin{itemize}
        \item Every interior points of \(\Omega\) is a limit point of \(\Omega\).
        \item \(x_0\) is not necessarily in \(Omega\)
        \item A set is closed \(\Leftrightarrow\) it contains all of its limit points.
    \end{itemize}
    
    \paragraph{Variations of a Set}
    Consider the set \(\Omega \in \mathbb{R}^n\).
    \begin{itemize}
        \item The \underline{interior} of \(\Omega\) is the set of all its interior points.
        \item The \underline{boundary} \(\partial \Omega)\) of \(\Omega\) is the set of all its boundary points.
        \item The \underline{closure} of \(\Omega\): \(\bar{\Omega} = \Omega \cup \partial \Omega\).
    \end{itemize}
    The interior is the largest open subset and the closure is the smallest closed set containing \(\Omega\).

    \paragraph{Limit of a Function at a Point}
    For \(f: \Omega \subset \mathbb{R}^n \to \mathbb{R}^m\), 
    \(\lim_{x\to x_0}\) means that 
    \begin{align*}
        \forall \epsilon \exists \delta > 0\text{ such that for } x\in\Omega: \\
        0 < d(x, x_0) < \delta \implies d(f(x), b) < \epsilon.
    \end{align*}
    Alternatively,
    \[
        x\in B(x_0, \delta) \backslash \{x_0\}
        \implies f(x)\in B(b, \epsilon).
    \]
    It is sufficient to consider the limits of the components of a function.

    \paragraph{Limits and sequences}
    The limit \(\lim_{x\to a} f(x) = b\) exists if and only if,
    \(\lim_{k\to\infty} f(x_k) = b\)  for all sequences \({x_k}\) such that 
    \(x_k\) is an element of \(\Omega\) and,
    \(\lim_{k\to\infty} x_k = a\).
    
    This is very helpful for showing that a limit does not exists.

    \subsection{Pinching and IVT Theorem}
    \paragraph{Pinching Theorem}
    \paragraph{IVT}
    see 1141
    
    
    % TODO: #1 Analysis Notes
    \section{Analysis}
    % TODO: #2 Differentiation Notes
    \section{Differentiation}
    % TODO: #3 Integration Notes
    \section{Integration}
    
    % 
    % 
    % 


    \section{Fourier Series}
    \paragraph{Fourier Series}
    A Fourier series is the approximation of simple periodic functions by
    the sum of period functions of the form \(\sin(x), \cos(x)\).
    Note that unlike Taylor series, a function \(f\) may be discontinuous.
    However, any lack of continuity leads to an infinite sum in the Fourier series.
    
    \subsection{Inner Products}
    \paragraph{Inner Products}
    Let \(V\) be a real vector space. An inner product on \(V\) is a map
    that assigns each \(f,g\in V\) a real number \(\langle f, g \rangle\)
    such that the following properties hold for all \(f, g, h \in V\) and\
    \(\lambda, \mu \in \mathbb{R}\):
    \begin{itemize}
        \item \(\langle f, f \rangle \geq 0\),
        \item \(\langle f, f \rangle = 0\) if and only if \(f\) is zero,
        \item \(\langle \lambda f + \mu g, h\rangle\),
        = \(\lambda\langle f, h\rangle\) + \(\mu\langle g, h\rangle\),
        \item \(\langle g, f \rangle = \langle f, g \rangle\).
    \end{itemize}

    \paragraph{Usual Inner Products}
    \begin{itemize}
        \item The vector space \(R^n\) admits the following inner product
        \[ 
            \langle u, v \rangle = u\cdot v = \sum_{i=1}^n u_i v_i.
        \]
        \item The vector space \(C[a, b]\) consisting of all continuous
        function on the interval \([a, b]\) admits the following inner product
        \[
            \langle f, g \rangle = \int_{a}^{b} f(x) g(x) dx.
        \]
    \end{itemize}

    \paragraph{Inner Product and Orthogonality}
    We say functions are orthogonal if \(\langle f, g\rangle = 0\).
    
    \subsection{Norms}
    A norm on \(V\) is a map that assigns each \(f\in V\) a real number
    \(||f||\) such that \(\forall f \in V, \lambda \in \mathbb{R}\)
    \begin{itemize}
        \item \(||f|| > 0\),
        \item \(||f|| = 0\) if and only if \(f = 0\),
        \item \(||\lambda f|| = \lambda ||f||\),
        \item \(||f + g|| \leq ||f|| + ||g||\); that is, the triangle inequality holds.
    \end{itemize}
    
    \paragraph{Usual Norms}
    \begin{itemize}
        \item The Euclidian norm (\(L^2\)-norm): is a norm on \(C[a, b]\):
        \[||f||_2 = \sqrt{\int_a^b f(x)^2 dx)}\]
        \item The max norm is a norm on \(C[a, b]\):
        \[||f||_\infty = \max_{a \leq x \leq b} \{|f(x)|\}\]
    \end{itemize}
    
    \subsection{Fourier Coefficient and Series}

    \paragraph{Fourier Series}
    Suppose that a function \(f: \mathbb{R}\to \mathbb{R}\) is \(2L\)-periodic,
    - that is, \(f(x) = f(x+2L)\) - 
    and is square integrable - that is, \(\int_{-L}^{L}f(x)^2 dx < \infty\).
    Then, \(f\) may be represented by a Fourier series of the form
    \[
        f(x) =
        \frac{a_0}{2} + \sum_{k=1}^n
        \left[
            a_k \cos\left(\frac{k\pi}{L}x\right) + b_k \sin \left(\frac{k\pi}{L}x \right)
        \right]
        \quad
        \forall x\in [-\pi, \pi].
    \]
    This series converges to \(f\) as \(n\to \infty\).
    
    \paragraph{Fourier Coefficients}
    \begin{itemize}
        \item \(a_k = \frac{1}{L} \int_{-L}^{L} f(x) \cos\left( \frac{k\pi x}{L} \right)\)
        \item \(b_k = \frac{1}{L} \int_{-L}^{L} f(x) \sin\left( \frac{k\pi x}{L} \right)\)
    \end{itemize}
    
    \subsection{Convergence of Fourier Series}
    \paragraph{Continuity}
    Consider a function \(f: \mathbb{R}\to \mathbb{R}\) and a point \(c\in \mathbb{R}\).
    Suppose that the one-sided limits \(f(c^+)\) and \(f(c^-)\) exist.

    \begin{itemize}
        \item If \(f^{c^+} = f^{c^-} = f(c)\) then \(f\) is continuous at \(c\),
        \item If \(f^{c^+} = f^{c^-} \neq f(c)\) then \(f\) has a removable discontinuity at \(c\),
        \item If \(f(c^+) \neq f(c^-)\) then, \(f\) has a jump discontinuity at
        at c.
    \end{itemize}

    \paragraph{Piecewise Continuity}
    A function is piecewise continuous on \([a, b]\) if and only if 
    \begin{itemize}
        \item \(f(x^+)\) exists \(\forall x\in [a, b]\),
        \item \(f(x^-)\) exists \(\forall x\in [a, b]\),
        \item \(f\) is continuous on \((a, b)\) except at most a finite number of points.
    \end{itemize}
    Note that if \(f\) is only piecewise continuous then the partial sum of the Fourier series does not necessarily converge to \(f\) for all \(x\).

    \paragraph{Piecewise differentiability}
    A function \(f\) is differentiable on \(c\) if and only if
    \(f(c^+) = f(c^-) = f(c)\) and \(D^+ f(c) = D^- f(c)\)
    
    Note: \(D^+ f(c)\) is not necessarily the same as \(\lim_{x\to c^+}f'(x)\).

    A function is piecewise differentiable on \([a, b]\) if and only if 
    \begin{itemize}
        \item \(D^+f(x^)\) exists \(\forall x\in [a, b)\),
        \item \(D^-f(x^)\) exists \(\forall x\in (a, b]\),
        \item \(f\) is differentiable on \((a, b)\) except at most a finite number of points.
    \end{itemize}

    \paragraph{Pointwise convergence}
    Let \(c\in \mathbb{R}\). Suppose that a function has the following properties

    \begin{itemize}
        \item \(f\) is \(2L\) periodic,
        \item \(f\) is piecewise continuous on \([-L, L]\),
        \item \(D^+f(c), D^-f(c)\) exist.
    \end{itemize}
    
    Then,
    \[
    S_f(c) = \frac{1}{2} [f(c^+) + f(c^-)].
    \]
    Observe that if \(f\) is continuous at \(c\) then \(S_f(c) = f(c)\).

    \paragraph{Odd and Evenness} 
    Recall that odd and even functions are defined by the conditions \(f(-x) = -f(x)\) and \(f(x) = f(-x)\) respectively.

    The following elementary properties hold:
    \begin{itemize}
        \item Odd \(\times\) Even \(=\) Even,
        \item Odd \(\times\) Odd \(=\) Even,
        \item Even \(\times\) Even \(=\) Even,
        \item \(\int_{-L}^L Odd = 0\).
    \end{itemize}

    \subsection{Convergence of Sequences}
    
    \paragraph{Pointwise convergence}
    Let \(f_k: \mathbb{R} \to \mathbb{R}\). \(f_k\) converges to \(f\) on
    \([a, b]\) pointwisely iff and only if for all \(x\in[a,b]\),
    \(f_k(x)\to f(x)\) as \(k\to infty\).

    \paragraph{Epsilon Delta Definition Pointwise Convergence}
    For all \(x\in [a,b]\), \(\epsilon > 0, \exists K\) (which will depend on \(\epsilon, x\) such that
    \[|f_k(x) - f(x)| \leq \epsilon \quad\forall k \geq K.\]

    \paragraph{Uniform Convergence}
    Let \(f_k: \mathbb{R}\to \mathbb{R}\). \(f_k\) converges to \(f\) on [a, b] uniformly if and only if
    for all \(\epsilon > 0, \exists K\) (depending on \(\epsilon\) only) such that
    \[
    \sup_{x\in [a,b]} |f_k(x) - f(x)| \leq \epsilon \quad\forall k \geq K.
    \]

    \paragraph{Weierstrass test}
    Let \(f_k: \mathbb{R}\to \mathbb{R}\) be a sequence of a function \(f\) defined on \([a, b]\).
    Suppose that there exists a sequence of numbers \(c_k\) such that
    \[|f_k(x)| \leq c_k \quad \forall x\in [a, b]\]
    where \(\sum_{k=1}^\infty c_k\) converges to a real number.
    Then
    \(\sum_{k=1}^\infty f_k\) converges uniformly to a function \(f\) on \([a, b]\).

    Note that this test also holds for function \(f: \mathbb{R}^n \to \mathbb{R}\) for \(x\in \Omega\) where \(\Omega\) is a closed bounded set in \(\mathbb{R}^n\).

    \paragraph{Norm Convergence}
    Using the supremum norm, the definition of uniform convergence can be equivalently written as:
    for all \(\epsilon > 0, \exists K\) such that
    \[
        ||f_k - f|| \leq \epsilon \quad \forall k \geq K.
    \]
    Equivalently,
    \[\lim_{k\to\infty} ||f_k - f|| = 0.\]
    We may extend this to define norm-convergence for any norm.

    \paragraph{Extending Norm Convergence to L-2}
    Recall from the previous paragraph that norm-convergence is defined as follows:
    \[\lim_{k\to\infty} ||f_k - f|| = 0.\]
    As such, \(L^2\) norm convergence, also knows as mean square convergence
    is equivalent to the following
    \[\lim_{k\to\infty} \int_a^b [f_k(x)-f(x)]^2 dx = 0.\]

    \paragraph{Parseval Theorem}
    Let \(f\) be a \(2\pi\) periodic and bounded function where
    \(\int_{-pi}^\pi f(x)^2 dx < +\infty\). Then, the Fourier series of \(f\) converges to \(f\) in the mean square sense. Moreover, the Parseval's
    identity holds
    \[
    \int_{-pi}^\pi f(x)^2 = ||f||_2^2 = \frac{\pi}{2}a_0 +
    \pi \sum_{k=1}^\infty  (a_k^2 +b_k^2).
    \]
    This identity continues to hold for \(2L\) periodic functions integrated
    over \([-L, L]\).

    % TODO: #8 Look at heat equation
    
    
    \section{Vector Fields}

    \subsection{Vector Fields and Flows, Divergence and Curl}
    \paragraph{Flow Lines}
    If \(F\) is a vector field, a flow line for \(F\) is a path \(c(t)\)
    such that
    \[c'(t) = F(c(t)).\]
    That is, that \(F\) yields the velocity field of the path \(c(t)\).

    \paragraph{The Del \(\nabla\) operator}
    The vector differential operator \(\nabla\) may be considered a 
    symbolic vector.
    The differential operator may be written as 
    \[\nabla = \frac{\partial}{\partial x}i + \frac{\partial}{\partial y}j + \frac{\partial}{\partial z}k.\]

    \paragraph{Divergence}
    Simply put, \[\text{div} F = \div F.\]

    Divergence may be thought as a type of derivative that describes the measure at which 
    a vector field \textit{spreads away} from a certain point.
    If the divergence is positive, then there is a net outflow while there is net inflow if the divergence is negative.

    Observe that the divergence of a vector field will be real-valued.

    \paragraph{Curl}
    If \(F\) is a vector field, then the curl may be defined as
    \[\mathrm{curl}F = \curl F.\]

    Curl is also analogous to a type of derivative for vector fields. The curl may be though as the measure at which the vector field \textit{swirls} around a point.
    A positive swirl can be thought of as a counter clockwise rotation.

    Observe that the curl of a vector field is also a vector field.

    % TODO #9 Add vector identities
    
    \section{Path Integrals}

    \subsection{Path Integrals}

    \paragraph{Path (Scalar Line) Integrals}
    Suppose that a vector-valued function \(c(t)\) parametrises a 
    curve \(C\) for \(t\in [a, b]\).
    The scalar line integral may be thought as the integral of along \(c\).

    \paragraph{Computing a Scalar Line Integral}
    Let \(c(t)\) parametrise a curve \(C\) for \(t\in [a, b]\).
    Assume that \(f(x, y, z)\) and \(c(t)\) are continuous.
    Then,
    \[
        \int_{C} f(x, y, z) ds
        =
        \int_{a}^{b} f\left(c\left(t\right)\right) \cdot ||c(t)|| dt.
    \]

    \paragraph{Elementary Properties of Path Integrals}
    \begin{itemize}
        \item \(\int_C f_1 ds + \int_C f_2 ds = \int_C(f_1+f_2)ds\),
        \item \(\int_C\lambda f ds = \lambda \int_C f ds, \quad
        \lambda\in \mathbb{R}\).
    \end{itemize}
    
    \subsection{Applications Of Path Integrals}
    Suppose that \(\delta = \delta(x, y, z)\) which is a density function.

    \paragraph{Mass}
    \[M = \int_C \delta(x, y, z) ds.\]

    \paragraph{First Moments About the Coordinate Plane}
    \begin{itemize}
        \item \(M_{yz} = \int_C x \delta ds\)
        \item \(M_{xz} = \int_C y \delta ds\)
        \item \(M_{xy} = \int_C z \delta ds\)
    \end{itemize}
    
    \paragraph{Coordinates of Center of Mass}  
    \begin{itemize}
        \item \(\bar{x} = \frac{M_{yz}}{M}\)
        \item \(\bar{y} = \frac{M_{xz}}{M}\)
        \item \(\bar{z} = \frac{M_{xy}}{M}\)
    \end{itemize}
    
    \paragraph{Moments of Inertia About Axes}
    \begin{itemize}
        \item \(I_x = \int_C (y^2 + z^2) \delta ds\)
        \item \(I_x = \int_C (x^2 + z^2) \delta ds\)
        \item \(I_x = \int_C (x^2 + y^2) \delta ds\)
    \end{itemize}
    
    
    \section{Vector Line Integrals}
    \paragraph{Vector Line Integrals}  
    Vector line integrals are different from scalar line integrals in the
    sense that to define a vector line integral, we must specify a direction
    along the path or curve \(C\).

    \paragraph{Computing a Vector Line Integral}
    Let \(c(t)\) parametrise an oriented curve \(C\) for \(t\in [a, b]\).
    Then, 
    \[
        \int_C F\cdot ds = \int_a^b F(c(t)) \cdot c'(t).
    \]
    
    \paragraph{Link to Path Integrals}
    Suppose that \(C\) is a smooth curve with a parametrisation \(c(t)\)
    for \(t\in[a, b]\) where \(c(t)\) is continuously differentiable and
    \(c'(t) \neq 0\) for all \(t\in [a, b]\).

    Then, \(c'(t)\) is a non-zero tangent vector pointing in the forward direction and the unit tangent vector is 
    \[T(c(t)) = \frac{c'(t)}{||c'(t)||}.\]
    Then,
    \[
        \int_C F\cdot ds = \int_C F \cdot T ds.
    \]
    
    \paragraph{Summing Paths}
    Suppose that \(C\) is made of \(n\) finitely many paths \(C_i\).
    Then, \(C = \sum_{i}^n C_i\).
    Note that all the curves must be joined end to end.
    Then,
    \[\int_C F \cdot ds = \sum_{i}^n \int_{C_i} F \cdot ds.\]

    \paragraph{Work and Other Alternative Notations}
    Suppose that \(c(t) = (x(t), y(t), z(t))\) and
    \(F = (M, N, P)\).
    Then, we denote work as any of the following notations
    \begin{align*}
        W
        &= \int_C F\cdot ds \\
        &= \int_a^b \left(
            M \frac{dx}{dt} + N \frac{dy}{dt} + P \frac{dz}{dt}
        \right) dt \\
        &= \int_C M ds + N ds + P ds.
    \end{align*}
    
    \paragraph{Properties fo Line Integrals}
    % TODO - 
    \begin{itemize}
        \item Linearity
        \item Reversing Orient
    \item Additivity
    \end{itemize}

    \paragraph{Flow Integrals and Circulation}
    Suppose that \(F\) represents a velocity field of a fluid flowing
    through a region in space. Then, the flow across a curve may be
    defined as the following
    \[
    \text{Flow} = \int_a^b F\cdot \hat{T}ds.
    \]
    This integral is called the flow integral. If the curve is a closed
    loop then this is called the \textit{circulation} around the curve.

    \paragraph{Flux in the Plane}
    If \(C\) is a smooth closed curve in the domain of a continuous vector field
    \(F = M(x, y)i + N(x, y)j + N(x, y)j\)
    and, \(n\) is the outward pointing unit-normal on \(C\) then, the flux
    of \(f\) across \(C\) is the following expression
    \[\int_C F\cdot \hat{n} ds.\]

    \paragraph{Calculating Flux Across a Smooth Closed Plane Curve}
    Suppose that \(F = Mi = Nj\). Let \(G = -N, M\)
    \[
    \text{Flux of \(F\) across \(C\)} = 
    \oint_C M dy - N dx = \oint_C G ds
    \]
    
    \subsection{Fundamental Theorem of Line Integrals}
    
    \paragraph{Gradient Fields}
    A vector field \(F\) is called a gradient vector field if there
    exists a real-valued function \(\phi\) such that
    \(F = \nabla \phi\). 
    That is,
    \[
        \begin{pmatrix} M \\ N \\ P \end{pmatrix}
        =
        \begin{pmatrix}
            \dfrac{\partial \phi}{\partial x} \\
            \dfrac{\partial \phi}{\partial y} \\
            \dfrac{\partial \phi}{\partial z}
        \end{pmatrix}.
    \]
    If such a function \(\phi\) exists then \(\phi\) is called the potential function of \(F\)
    where, \(F\) is \textit{conservative}.

    \paragraph{Fundamental Theorem for Gradient Vector Fields}
    If \(F = \nabla \phi\) on a domain \(\mathcal{D}\). Then,
    for all oriented curves \(C\) in \(\mathcal{D}\) with an initial
    point \(P\) and a terminal point \(Q\),
    \[
        \int_C F\cdot ds = \phi(Q) - \phi(P).
    \]
    The integral is independent of the path.
    
    \paragraph{Cross Partials of Gradient Vector Fields are Equal}
    Let \(F = (F_1, F_2, F_3)\) be a gradient vector field whose
    components have continuous partial derivatives.
    Then, the cross partials are equal. That is,
    \begin{align*}
        \frac{\partial F_1}{\partial y} &= \frac{\partial F_2}{\partial x},\\
        \frac{\partial F_2}{\partial z} &= \frac{\partial F_3}{\partial y},\\
        \frac{\partial F_3}{\partial x} &= \frac{\partial F_1}{\partial z}.
    \end{align*}
    Equivalently,
    \[\curl F = 0.\]

    \subsection{Green's Theorem}
    Green's Theorem connects double integrals and line integrals.

    
    \paragraph{Green's Theorem: Flux Divergence or Normal Form}
    Let \(D\) be a bounded simple region in \(\mathbb{R}^2\) with a nonempty interior
    whose boundary consists of a finite number of smooth curves.
    Let \(C\)be the boundary of \(D\) with a positive (counter-clockwise) direction.
    Let \(F=Mi+Nj\) be a continuously different boundary vector field on D.

    Then, the outward flux of \(F\) across the curve \(C\) equals the double
    integral of divergence \(\div F\) over \(D\). That is,
    \[
        \oint_C (F\cdot \hat{n})ds = \oint_C -Ndx + Mdy
        = \int\int_D  \left(\frac{\partial M}{\partial x} + \frac{\partial N}{\partial Y}\right).
    \]
    
    Once again, note the assumptions:
    \begin{itemize}
        \item \(D\) is bounded and simple with a non-empty interior,
        \item The boundary \(C\) is oriented in the positive (counter-clockwise) direction,
        and is the finite union of smooth curves,
        \item The vector field \(F\) is continuously differentiable on \(D\).
    \end{itemize}
    
    \paragraph{Green's Theorem: Circulation-Curl or Tangential Form}
    Let \(D\) be a bounded simple region in \(\mathbb{R}^2\) with a nonempty interior
    whose boundary consists of a finite number of smooth curves.
    Let \(C\)be the boundary of \(D\) with a positive (counter-clockwise) direction.
    Let \(F=Mi+Nj\) be a continuously different boundary vector field on D.
    
    Then, the counter-clockwise circulation of \(F\) around \(C\) equals the double
    integral of \((\curl F)\cdot k\) over \(D\).
    That is,
    \[
        \oint_C (F\cdot \hat{T}) ds =      
        \oint_C Mdx + Ndy = 
        \int\int_D  \left(\frac{\partial N}{\partial x} - \frac{\partial M}{\partial y}\right) dx dy.
    \]

    \paragraph{Area of a Region}
    Let \(D\) be a simple and bounded region with a non-empty interior and let \(C\)
    be the boundary of \(D\) which is the finite union of smooth curves.
    Then, the area of \(D\) can be calculated as such
    \[
    \text{Area}(D) = \frac{1}{2} \oint_C \left(-y dx + xdy\right).
    \]

    
    % % TODO: #6 Surface Integral Notes
    % \section{Surface Integrals}
    % % TODO: #7 Integral Theorems Notes
    % \section{Integral Theorems}

    
    
    
    \section{}
\end{document}
