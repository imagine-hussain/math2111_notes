\documentclass[12pt, letterpaper]{article}

\usepackage[margin=1in]{geometry}
\usepackage{amsmath}
\usepackage{amssymb}
\usepackage{mathtools}
\usepackage{fancyhdr}
\usepackage[utf8]{inputenc}
\usepackage{dirtytalk}                      % \say command for quotes
\usepackage{ wasysym }
\usepackage{graphicx}

\usepackage{hyperref}
\hypersetup {                                % Formatting for hyperlinks
    colorlinks,
    citecolor=black,
    filecolor=black,
    linkcolor=black,
    urlcolor=black
}

\graphicspath{'./assets/'}								% Path for images

\pagestyle{fancy}
\setlength{\headheight}{15pt}
\renewcommand{\headrulewidth}{0pt}
\renewcommand{\footrulewidth}{0pt}	
\title{Higher Several Variable Calculus \\ Math2111 UNSW}
\author{Hussain Nawaz \\ hussain.nwz000@gmail.com}
\date{2022T1}

\rhead{}
\lhead{}


\begin{document}
\maketitle
\tableofcontents
\newpage

    \section{Curves and Surfaces}
    \subsection{Curves}
    \paragraph{Curves}
    A curve in \(\mathbb{R}^n\) is a vector function
    \[\mathbf{c}: I \to\mathbb{R}^n,\]
    where \(I\) is an interval in \(\mathbb{R}\).

    \paragraph{Forms / Notations}
    Curves may be defined in the following ways:
    \begin{itemize}
        \item \textbf{Parametrically} by \(c(t) = \left(x_1(t), x_2(t), \dots, x_n(t)\right)\)
        \item \textbf{Cartesian} by eliminating the \(t\) variable to get \(y\) in terms of \(x\)
        \item \textbf{Implicitly} As \(F(x, y) = 0\).
    \end{itemize}

    % 
    % 
    % 
    % 
    % 
    % 
    % 
    % 
    % 
    \newpage

    \section{Analysis}

    \subsection{Assumed}
    \paragraph{Assumed Concepts from Real Single-Variable Calculus}
    \begin{itemize}
        \item limits
        \item continuity
        \item differentiability
        \item integrability
    \end{itemize}

    \paragraph{Assumed Theorems}
    \begin{itemize}
        \item Min/ Max Theorem
        \item Intermediate Value Theorem
        \item Mean Value Theorem
    \end{itemize}
    
    \subsection{Limits}
    Recall that \(\lim_{x\to a} f(x) = L\) requires that for all
    \(\epsilon > 0\), there exists a \(\delta > 0\) such that
    if \(|x-a| < \delta\)
    then
    \[|f(x) - L|  < \delta.\]
    
    
    
    \subsection{Metrics}
    We have metrics (distance functions) as
    \[m: \mathbb{R}^n\times\mathbb{R}^n \to \mathbb{R}\]
    satisfying the following 3 axioms.
    \begin{itemize}
        \item \textbf{Positive Definite} such that for all \(x,y\in\mathbb{R}^n\),
        \(m(x,y) > 0\) and, \(m(x,y) = 0\ \Leftrightarrow x = y\).
        \item \textbf{Symmetric} \(m(x,y) = m(y, x)\).z
        \item \textbf{Triangle Inequality} such that for all \(x,y,z\in\mathbb{R}^n\),
        \(m(x,y) + m(y, z) \leq m(x, z)\).
    \end{itemize}

    \paragraph{Euclidian Distance}
    We allow the Euclidian distance to be defined as
    \[d_n(x,y) := ||x-y|| = \sqrt{\sum_{i=1}^{n} (x_i - y_y)^2}\]
    We often allow \(d\) to be \(d_2\).

    \subsection{Limits of Sequences}
    \(//\text{soon}^\text{tm}\)


\end{document}